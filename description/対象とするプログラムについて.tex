\documentclass{jlreq}
\usepackage{graphicx}
\usepackage{listings,jvlisting} %日本語のコメントアウトをする場合jvlisting(もしくはjlisting)が必要
%ここからソースコードの表示に関する設定
\lstset{
  basicstyle={\ttfamily},
  identifierstyle={\small},
  commentstyle={\smallitshape},
  keywordstyle={\small\bfseries},
  ndkeywordstyle={\small},
  stringstyle={\small\ttfamily},
  frame={tb},
  breaklines=true,
  columns=[l]{fullflexible},
  numbers=left,
  xrightmargin=0,
  xleftmargin=3,
  numberstyle={\scriptsize},
  stepnumber=1,
  numbersep=1,
  lineskip=-0.5ex
}

\title{判定プログラムの入力の対象とするプログラムについて}
\author{谷元柊介}
\date{2024/11/12}

\begin{document}
    \maketitle
    \section{入力プログラムの概要}
        \subsection{入力プログラムとは}
        ここでいう「入力プログラム」は拡張子が.py であるようなPython ソースファイルのことである. Python ソースファイルとはPython のソースコードが記述されているファイルのことである.

        \subsection{Python ソースコードの構成}
        Python ソースコードは以下の要素のみで構成される\footnote{このことはPython 言語リファレンスに詳しく記述されている\cite{目次}.}.
        \begin{itemize}
            \item if やfor のような\textgt{複合文}
            \item break やreturn のような\textgt{単純文}
            \item コメント文
        \end{itemize}
        すなわち, Python ソースコードの構成要素は必ず複合文か単純文のどちらかに分類される.

    \section{複合文(compound statement)}
        \subsection{複合文とは}
            複合文とは, 1 つ以上の節から構成される文のことである\cite{目次}. ここでいう節はさらにヘッダとスイートの2 つから構成される.

            ヘッダは, if やwhile といった文法で指定されている一意に識別できるキーワードで始まり, コロンで終わる文のことである. スイートは節によって制御される文の集まりである\cite{目次}.
            
            \includegraphics[]{ヘッダスイート.png}    
                
        \subsection{本研究で扱う複合文}
            本研究では入力プログラムが複合文をもつ場合, その複合文は以下のいずれかであるとする\footnote{これ以外にも複合文とされる文は存在するが, 現段階では対象としない}.
            \begin{itemize}
                \item if 文とそれに伴うelif, else 文
                \item while 文とそれに伴うelse 文
                \item for 文とそれに伴うelse 文
                \item funcdef 文
            \end{itemize}
            すなわち, 複合文は以下のように定義される.
            \begin{lstlisting}[caption=式の定義]
                compound_stmt ::=
                    | function_def
                    | if_stmt
                    | for_stmt
                    | while_stmt
            \end{lstlisting}
            \subsubsection{if 文}
                本研究において, if 文は以下のように定義される.
                \begin{lstlisting}[caption=if 文の定義]
                    if_stmt ::=
                        | 'if' named_expression ':' block elif_stmt 
                        | 'if' named_expression ':' block [else_block] 
                    elif_stmt ::=
                        | 'elif' named_expression ':' block elif_stmt 
                        | 'elif' named_expression ':' block [else_block] 
                    else_block ::=
                        | 'else' ':' block
                    block ::=
                        | NEWLINE INDENT statements DEDENT 
                        | simple_stmts
                \end{lstlisting}
            \subsubsection{while 文}
            本研究においてwhile 文は以下のように定義される
            \begin{lstlisting}[caption=while 文の定義]
                while_stmt ::=
                    | 'while' named_expression ':' block [else_block]
            \end{lstlisting}
            \subsubsection{for 文}
            本研究においてfor 文は以下のように定義される.
            \begin{lstlisting}[caption=for 文の定義]
                for_stmt ::=
                    | 'for' star_targets 'in' ~ star_expressions ':' [TYPE_COMMENT] block [else_block]
            \end{lstlisting}
            \subsubsection{funcdef 文}
            本研究においてfuncdef 文は以下のように定義される
            \begin{lstlisting}[caption=funcdef 文の定義]
               function_def: 
                    | function_def_raw
                function_def_raw:
                    | 'def' NAME '(' [params] ')' ':' block 
                params:
                    | parameters
                parameters: 
                    | param_no_default+ param_with_default*  
                    | param_with_default+ 
                param_no_default:
                    | param ',' 
                    | param 
                param_with_default:
                    | param default ',' 
                    | param default 
                default: '=' expression
                param: NAME
            \end{lstlisting}
    \section{単純文(simple statement)}
        \subsection{単純文とは}
            単純文とは, 1 つの論理行に収められる文である\footnote{1 つの論理行に収まらない文だからといって, 必ずしも複合文ではない}. 
            
            論理行1 行は, 1 行以上の物理行から構成される. 物理行は行終端コードで区切られた文字列のことを指す.

        \subsection{本研究で扱う単純文}
            本研究では入力プログラムが単純文をもつ場合, その単純文は以下のいずれかであるとする\footnote{これ以外にも単純文とされる文は存在するが, 現段階では対象としない}.
            \begin{itemize}
                \item 式文(関数や$a+1$ のような式(expression)を文として呼び出すもの)
                \item 代入文
                \item pass 文
                \item break 文
                \item continue 文
            \end{itemize}
            すなわち単純文は以下のように定義される.
            \begin{lstlisting}[caption=式の定義]
                simple_stmt ::=
                    | assignment
                    | expression 
                    | 'pass' 
                    | 'break' 
                    | 'continue' 
            \end{lstlisting}
            \subsubsection{前提}
                以下, この先での文法の定義に際しての前提事項を述べる.
                \begin{itemize}
                    \item $e*$ はe というパターンが0 回以上連続することを示す.
                    \item $e+$ はe というパターンが1 回以上連続することを示す.
                    \item $e_1 \&e_2$ は$e_1$というパターンの中に, $e_2$というパターンが必ず含まれなければならないことを示す.
                    \item $e_1 !e_2$ は$e_1$というパターンの中に, $e_2$というパターンが含まれてはならないということを示す.
                \end{itemize}
            \subsubsection{式, 式文(expression)}
                本研究において, 式および式文は以下のように定義される\footnote{Python の本来の文法として式文に対応するものはstarred\_expression であるが, 本研究では簡略化のためstarred\_expression もexpressionも同一のものとして扱っている}
                \begin{lstlisting}[caption=式の定義]
                    expression ::= 
                        disjunction 'if' disjunction 'else' expression 
                        | disjunction
                    disjunction ::=
                        conjunction ('or' conjunction )+ 
                        | conjunction
                    conjunction ::=
                        inversion ('and' inversion )+ 
                        | inversion
                    inversion ::=
                        'not' inversion 
                        | comparison
                    comparison ::=
                        bitwise_or compare_op_bitwise_or_pair+ 
                        | bitwise_or
                    compare_op_bitwise_or_pair ::=
                        eq_bitwise_or
                        | noteq_bitwise_or
                        | lte_bitwise_or
                        | lt_bitwise_or
                        | gte_bitwise_or
                        | gt_bitwise_or
                        | notin_bitwise_or
                        | in_bitwise_or
                        | isnot_bitwise_or
                        | is_bitwise_or
                    eq_bitwise_or ::= '==' bitwise_or 
                    noteq_bitwise_or::=
                        ('!=' ) bitwise_or 
                    lte_bitwise_or ::= '<=' bitwise_or 
                    lt_bitwise_or ::= '<' bitwise_or 
                    gte_bitwise_or ::= '>=' bitwise_or 
                    gt_bitwise_or ::= '>' bitwise_or 
                    notin_bitwise_or ::= 'not' 'in' bitwise_or 
                    in_bitwise_or ::= 'in' bitwise_or 
                    isnot_bitwise_or ::= 'is' 'not' bitwise_or 
                    is_bitwise_or ::= 'is' bitwise_or
                    bitwise_or ::=
                         bitwise_or '|' bitwise_xor 
                        | bitwise_xor
                    bitwise_xor ::=
                        bitwise_xor '^' bitwise_and 
                        | bitwise_and
                    bitwise_and ::=
                        bitwise_and '&' shift_expr 
                        | shift_expr
                    shift_expr ::=
                        shift_expr '<<' sum 
                        | shift_expr '>>' sum 
                        | sum
                    sum ::=
                        sum '+' term 
                        | sum '-' term 
                        | term
                    term ::=
                        term '*' factor 
                        | term '/' factor 
                        | term '//' factor 
                        | term '%' factor 
                        | term '@' factor 
                        | factor
                    factor ::=
                        '+' factor 
                        | '-' factor 
                        | '~' factor 
                        | power
                    power ::=
                        await_primary '**' factor 
                        | primary
                    primary ::=
                        | primary '.' NAME 
                        | primary '(' [arguments] ')' 
                        | primary '[' named_expression ']' 
                        | atom
                    named_expression:
                        | expression !':='


                    atom ::=
                        NAME
                        | 'True' 
                        | 'False' 
                        | 'None' 
                        | strings
                        | NUMBER
                        | (tuple | group)
                        | list
                        | (dict | set)
                        | '...'
                \end{lstlisting}
                NAME とは, 変数名, 関数名, クラス名, モジュールやパッケージ名のことを指す. NUMBER はint やfloat といったデータ型のリテラルのことを指す.
            \subsubsection{引数(arguments)}
                本研究において, 引数は以下のように定義される.
                \begin{lstlisting}[caption=引数の定義,label=fuga]
                    arguments ::= args [','] &')'
                    args ::= ','.(expression !':=')+
                \end{lstlisting}
            \subsubsection{代入文(assignment statement)}
                本研究において, 代入文は以下のように定義される
                \begin{lstlisting}[caption=代入文の定義,label=fuga]
                    assignment_stmt ::=  
                        (star_atom "=")+ (expression)
                        | star_atom augassign ~ expression
                    augassign ::=
                        | '+=' | '-=' | '*=' | '@=' | '/=' | '%=' | '&=' | '|=' | '^=' | '<<=' | '>>=' | '**=' | '//='
                    star_atom ::= NAME
                \end{lstlisting}
            \subsubsection{pass 文, break 文, continue 文}
                これらの文には特有の構文要素はなく, "pass", "break", "continue"といったその文字列単体をもって文とみなされる.
    \begin{thebibliography}{temp}
    \bibitem{目次}
    Python 言語リファレンス(2024/11/12)
    
    https://docs.python.org/ja/3/reference/index.html
    \end{thebibliography}
\end{document}
